\documentclass[fontsize=12pt, % Document font size
                             paper=a4, % Document paper type
                             oneside, % Shifts odd pages to the left for easier reading when printed, can be changed to oneside
                             captions=tableheading,
                             index=totoc,
                             hyperref]{labbook}
 
\usepackage[bottom=10em]{geometry} % Reduces the whitespace at the bottom of the page so more text can fit
%\usepackage{media9}
%\usepackage[dvipdfmx]{media9} 
%\usepackage[svgnames]{xcolor}
\usepackage[english]{babel} % English language
\usepackage{lipsum} % Used for inserting dummy 'Lorem ipsum' text into the template


%\usepackage[utf8]{inputenc} % Uses the utf8 input encoding
\usepackage[T1]{fontenc} % Use 8-bit encoding that has 256 glyphs
%\usepackage{lmodern}
%\usepackage{fontspec}
\usepackage{emerald}
%usepackage{va}
%\usepackage{frcursive}
%\usepackage{calligra}

%\setmainfont{\ECFTeenSpirit}

%\renewcommand{\fontfamily{emerald}}{\ECFTeenSpirit}

%\renewcommand{\rmdefault}{emerald}
%\renewcommand{\familydefault}{pag}

%\usepackage[osf]{mathpazo} % Palatino as the main font
%\linespread{1.05}\selectfont % Palatino needs some extra spacing, here 5% extra
%\usepackage[scaled=.88]{beramono} % Bera-Monospace
%\usepackage[scaled=.86]{berasans} % Bera Sans-Serif

\usepackage{booktabs,array} % Packages for tables

\usepackage{amsmath} % For typesetting math

\newcommand{\fullpage}[1]{
\begin{frame}
 #1
\end{frame}
}
\usepackage{movie15}  

\usepackage{etoolbox}
\usepackage[norule]{footmisc} % Removes the horizontal rule from footnotes
\usepackage{lastpage} % Counts the number of pages of the document
\usepackage{color}
\usepackage{listings}

\definecolor{Code}{rgb}{0,0,0}
\definecolor{Decorators}{rgb}{0.5,0.5,0.5}
\definecolor{Numbers}{rgb}{0.5,0,0}
\definecolor{MatchingBrackets}{rgb}{0.25,0.5,0.5}
\definecolor{Keywords}{rgb}{0,0,1}
\definecolor{self}{rgb}{0,0,0}
\definecolor{Strings}{rgb}{0,0.63,0}
\definecolor{Comments}{rgb}{0,0.63,1}
\definecolor{Backquotes}{rgb}{0,0,0}
\definecolor{Classname}{rgb}{0,0,0}
\definecolor{FunctionName}{rgb}{0,0,0}
\definecolor{Operators}{rgb}{0,0,0}
\definecolor{Background}{rgb}{0.98,0.98,0.98}



%\usepackage{helvet}
\renewcommand{\familydefault}{\sfdefault}

\usepackage[dvipsnames]{xcolor}  % Allows the definition of hex colors
\definecolor{titleblue}{rgb}{0.16,0.24,0.64} % Custom color for the title on the title page
\definecolor{linkcolor}{rgb}{0,0,0.42} % Custom color for links - dark blue at the moment
%\setkomafont{main}{\ECFAugie}
%\renewcommand*{\familydefault}{\normalfont}
%\setkomafont{}{\normalfont\ECFAugie}
%\renewcommand*{\seriesdefault}{\ECFAugie}
%\addtokomafont{hyperref}{\color{Sepia}}
\addtokomafont{title}{\ECFAugie\color{Mahogany}} % Titles in custom blue color
\addtokomafont{chapter}{\ECFAugie\color{LimeGreen}} % Lab dates in olive green
\addtokomafont{section}{\ECFAugie\color{Cyan}} % Sections in sepia
\addtokomafont{subsection}{\ECFAugie\color{LimeGreen}} % Sections in sepia
\addtokomafont{subsubsection}{\ECFAugie\color{Orange}} % Sections in sepia
\addtokomafont{pagehead}{\ECFAugie\color{Red}} % Header text in gray and sans serif
\addtokomafont{caption}{\ECFAugie\footnotesize\itshape} % Small italic font size for captions
\addtokomafont{captionlabel}{\ECFAugie\upshape\bfseries} % Bold for caption labels
\addtokomafont{descriptionlabel}{\ECFAugie}
\setcapwidth[r]{10cm} % Right align caption text
\setkomafont{footnote}{\sffamily} % Footnotes in sans serif

\deffootnote[4cm]{4cm}{1em}{\textsuperscript{\thefootnotemark}} % Indent footnotes to line up with text

\DeclareFixedFont{\textcap}{T1}{fjd}{bx}{n}{1.5cm} % Font for main title: Helvetica 1.5 cm
\DeclareFixedFont{\textaut}{T1}{fjd}{bx}{n}{0.8cm} % Font for author name: Helvetica 0.8 cm

\usepackage[nouppercase,headsepline]{scrpage2} % Provides headers and footers configuration
\pagestyle{scrheadings} % Print the headers and footers on all pages
\clearscrheadfoot % Clean old definitions if they exist

\automark[chapter]{chapter}
\ohead{\headmark} % Prints outer header

\setlength{\headheight}{25pt} % Makes the header take up a bit of extra space for aesthetics
\setheadsepline{.4pt} % Creates a thin rule under the header
\addtokomafont{headsepline}{\ECFAugie\color{lightgray}} % Colors the rule under the header light gray

\ofoot[\normalfont\normalcolor{\thepage\ |\  \pageref{LastPage}}]{\normalfont\normalcolor{\thepage\ |\  \pageref{LastPage}}} % Creates an outer footer of: "current page | total pages"

% These lines make it so each new lab day directly follows the previous one i.e. does not start on a new page - comment them out to separate lab days on new pages
\makeatletter
\patchcmd{\addchap}{\if@openright\cleardoublepage\else\clearpage\fi}{\par}{}{}
\makeatother
\renewcommand*{\chapterpagestyle}{scrheadings}

% These lines make it so every figure and equation in the document is numbered consecutively rather than restarting at 1 for each lab day - comment them out to remove this behavior
\usepackage{chngcntr}
\counterwithout{figure}{labday}
\counterwithout{equation}{labday}

\definecolor{dkgreen}{rgb}{0,0.6,0}
\definecolor{gray}{rgb}{0.5,0.5,0.5}
\definecolor{mauve}{rgb}{0.58,0,0.82}
 
\lstset{ %
  language=C++,                % the language of the code
  basicstyle=\footnotesize\ttfamily,           % the size of the fonts that are used for the code
  numbers=left,                   % where to put the line-numbers
  numberstyle=\tiny\color{gray},  % the style that is used for the line-numbers
  stepnumber=1,                   % the step between two line-numbers. If it's 1, each line 
                                  % will be numbered
  numbersep=5pt,                  % how far the line-numbers are from the code
  backgroundcolor=\color{black},      % choose the background color. You must add \usepackage{color}
  showspaces=false,               % show spaces adding particular underscores
  showstringspaces=false,         % underline spaces within strings
  showtabs=false,                 % show tabs within strings adding particular underscores
  frame=single,                   % adds a frame around the code
  rulecolor=\color{black},        % if not set, the frame-color may be changed on line-breaks within not-black text (e.g. comments (green here))
  tabsize=2,                      % sets default tabsize to 2 spaces
  captionpos=b,                   % sets the caption-position to bottom
  breaklines=true,                % sets automatic line breaking
  breakatwhitespace=false,        % sets if automatic breaks should only happen at whitespace
  title=\lstname,                   % show the filename of files included with \lstinputlisting;
                                  % also try caption instead of title
  keywordstyle=\color{dkgreen},          % keyword style
  commentstyle=\color{dkgreen},       % comment style
  stringstyle=\color{mauve},         % string literal style
  escapeinside={\%*}{*)},            % if you want to add LaTeX within your code
  morekeywords={Rect,...},              % if you want to add more keywords to the set
  deletekeywords={...}              % if you want to delete keywords from the given language
}

%Hyperlink configuration
\usepackage[
    pdfauthor={pi19404}, % Your name for the author field in the PDF
   % pdftitle={Article}, % PDF title
   % pdfsubject={}, % PDF subject
    %bookmarksopen=true,
    %linktocpage=true,
    urlcolor=Goldenrod, % Color of URLs
    citecolor=Goldenrod, % Color of citations
    linkcolor=Goldenrod, % Color of links to other pages/figures
    %backref=page,
    pdfpagelabels=true,
    plainpages=false,
    colorlinks=true, % Turn off all coloring by changing this to false
    bookmarks=true,
    pdfview=FitB]{hyperref}

%\usepackage[stretch=10]{microtype} % Slightly tweak font spacing for aesthetics

%\usepackage{pdfpages}                    % prikaz \includepdf
%\usepackage{attachfile}
%\usepackage{embedfile}
%\usepackage{graphicx}
\usepackage{caption}
\usepackage{subcaption}
\usepackage{movie15}
%\usepackage{media9}
%\usepackage[sort,sectionbib,square,round,authoryear]{natbib}
%\usepackage[sort]{cite}

%\usepackage{natbib}
\usepackage{filecontents}
\usepackage[natbib=true,citestyle=authoryear,bibstyle=numeric]{biblatex} 
%pdflatex dtw.tex;bibtex dtw.aux;pdflatex dtw.tex
\usepackage{graphicx}
% declare known graphics extensions
\DeclareGraphicsExtensions{.jpg,.jpeg,.pdf,.png,.mps,.gif}
\newcommand{\vurl}[1]{\url{#1}}
\usepackage{animate}
\usepackage{eso-pic}
\newcommand\BackgroundPic{
\put(0,0){
\parbox[b][\paperheight]{\paperwidth}{%
\vfill
\centering
\includegraphics[width=\paperwidth,height=\paperheight]{../back.jpg}%
\vfill
}}}
\usepackage{graphicx}


\usepackage{animate}
\renewcommand*{\compcitedelim}{\addsemicolon\space}
%\setlength\parindent{0pt} % Uncomment to remove all indentation from paragraphs
\usepackage{minted}
\ECFAugie\color{lightgray}
\bibliography{test} 
\begin{document}
\AddToShipoutPicture{\BackgroundPic}
%----------------------------------------------------------------------------------------
%	TITLE PAGE
%----------------------------------------------------------------------------------------
\ECFAugie\color{lightgray}
\title{\textcap{Mean Shift Tracking\\[2cm]  
%\textaut{Beginning 30-05-2012}
}}
\author{
    \textaut{Pi19404}\\ \\ % Your name
    %Master of Science % Your degree
}
%\ECFJD{\today} % No date by default, add \today if you wish to include the publication date

\maketitle % Title page
\printindex
\tableofcontents % Table of contents
\newpage % Start lab look on a new page
%
%\begin{addmargin}[0cm]{0cm} % Makes the text width much shorter for a compact look
%
\pagestyle{scrheadings} % Begin using headers
%
\ECFAugie\color{White}

\labday{Mean Shift Tracking}

%\ECFTeenSpirit
%\section{Augie} {\ECFJD\lipsum[4]}
%\section{Auriocus Kalligraphicus} {\Fontauri\lipsum[4]}
%\section{BrushScriptX-Italic} {\bsifamily\lipsum[4]}

\section{Introduction}
In the article we will look at the application of Mean Shift Tracking for color based
tracking.
\section{Mean shift}
The object model used in mean shift tracking  is color probability distribution.
\\\\
Now we have a object model,given an image we can compute the likelihood image
Each pixel in likelihood image represents the likelihood that pixel belongs
to the object model/histogram.
\[
 L_u(y_i) = p_u*\delta(b(y_i)-u)
\]
\begin{figure}[!htbp]
\begin{subfigure}[b]{0.3\textwidth}
                \centering
                \includegraphics[width=\textwidth]{image016.png}
                \caption{original image}                
        \end{subfigure}%
  \begin{subfigure}[b]{0.3\textwidth}
                \centering
                \includegraphics[width=\textwidth]{image015.png}
                \caption{likelihood image}                
        \end{subfigure}%
  \begin{subfigure}[b]{0.3\textwidth}
                \centering
                \includegraphics[width=\textwidth]{image017.png}
                \caption{Hue histogram}                
        \end{subfigure}%

        \caption{Object model}\label{fig:image1}
\end{figure}
This likelihood image assigns to each pixel a similarity measure wrt object model.
\\\\
It is reasonable to assume that the region in which the highest similarity measure 
or highest density is observed is a good estimate of object location.
\\\\
\begin{minted}[linenos=true,
mathescape,
numbersep=5pt,
frame=single,
framesep=2mm
]{c++}
//building the object appearance model
void ocvmeanShift::buildModel(Mat image,Rect rect)
{
    //input region of interest
    region=rect;
    //center of region is current location estimate
    p.x=region.x+region.width/2;
    p.y=region.y+region.height/2;
    //extract ROI
    Mat roi=image(rect);    
    //compute the histogram ,h is object of type Histogram
    h.BuildHistogram(roi);
}

//call to compute the likelihood after computing the model
Mat sim=h.likeyhoodImage(image);
    
\end{minted}
Thus if we consider a small window and move towards the mean value
ie along the mean shift vector we should eventually reach the region of maximum similarity.
\\\\
The  likelihood surface is not smooth,we can give it properties
of smoothness using kernel density estimation.
\\\\
Now we can find the modes of the similarity surface using standard
mean shift algorithm.
\\\\
Let us assume that current estimate of mode of function is at $y$.
Thus we consider a small rectangular window about $y$,compute
the mean shift vector and take a small step along the mean shift vector.
\\\\
In principle this should enable to find the local maximum.
Similarity surface is discontinuous and to do this we need to perform KDE over entire image
consisting of dense grid of points ,which is a  expensive operation.(We need to perform
convolution at each point with Gaussian with suitably large aperture)
\section{Using Similarity for tracking}
The concept of similarity surface can be made useful in tracking application.
\\\\
Let us consider a small region of interest about a present location y,
we can compute the similarity score about this region,perform KDE
on this small region,obtain a similarity surface and compute the mean shift vector.
\\\\
If object is not present in region,similarity surface will be flat and mean shift
vector will be zero.
\\\\
If there is object present in some part of region,it will correspond
to modes of similarity surface .The mean shift vector will give us
direction to move along.
\\\\
Now instead of trying to estimate the mode,say we translate the region
of interest along direction provided by the mean shift vector.
This would typically lead to large portion of object being visible
and would expose the region  of global similarity surface where
a large maximum would lie.
\\\\
\ECFAugie\color{Orange}
This is the basis of mean  shift tracking,keen on translating the region
of interest ,till we reach local maximum of similarity surface.
\ECFAugie\color{White}
\\\\
For tracking applications ,since fast computation is required,
we can consider a rectangular window with bandwidth equal to that
of the region of interest.The present location of point point
is the center of the rectangular region.
\subsection{Implementation}
\begin{minted}[linenos=true,
mathescape,
numbersep=5pt,
frame=single,
framesep=2mm
]{c++}
//compute the likelihood
Mat sim=h.likeyhoodImage(image);
//perform iteratively till convergence
    for(int i=0;i<criteria.maxCount;i++)
    {
    //extracting the region of interest
    Mat roi=sim(region);
    //compute the moments
    cv::Moments m;
    m=cv::moments(roi,false);
    
    //threshold m00 which is weighted mean,
    //exit since no similar pixels present
    if(fabs(m.m00)<region.width*region.height*0.05)
        break;

    //computing the mean values
    int x=cvRound(m.m10/m.m00);
    int y=cvRound(m.m01/m.m00);
        
    //computing the mean shift
    int dx=region.width/2-x;
    int dy=region.height/2-y;

    //displacement from current position
    int nx=p.x-dx;
    int ny=p.y-dy;

    //bounday of the image
    if(nx-region.width/2<=0) nx=region.width/2;
    if(nx+region.width/2>=image.cols) nx=image.cols-region.width/2-1;
    if(ny-region.height/2<=0) ny=region.height/2;
    if(ny+region.height/2>=image.rows) ny=image.rows-region.height/2-1;

        //recalculating the mean shift
    dx=-nx+p.x;
    dy=-ny+p.y;

    //checking magnitude of mean shift vector.

    float mag=dx*dx+dy*dy;

    //no change in mean,reached local maxima
    if(mag<criteria.epsilon*criteria.epsilon)
        break;

   //updating the position
    p.x=nx;
    p.y=ny;
    //updating the region of interest
    region.x=p.x-region.width/2;
    region.y=p.y-region.height/2;

    }
     
\end{minted}
A video of mean shift tracking is shown below.A naive object model based
on color probability in HS color space using first frame of the video
\vurl{https://googledrive.com/host/0B-pfqaQBbAAtNkg2bUJvWERmNFU/video00001.ogv}
\\\\
As will all tracking approaches ,the performance heavily depends on the object model.
The better we are able to model the object and obtain a likelyhood/similarity which
does not show high probability for background or other objects in the scene,the more 
accurate will be the tracking
\section{Code}
For further image processing application a library consisting of high level
interface to opencv will be used.The library is called OpenVisionLibrary.
\url{https://github.com/pi19404/OpenVision}
\\\\
The project cmake file is included in the repository.
the build will create the library and test files in the bin directory
To run demo program for mean shift run the binary \texttt{meanShiftTest}
\\\\
The files for mean shift algorithm are meanshift.cpp and meanshift.hpp
\url{https://github.com/pi19404/OpenVision/tree/master/ImgProc/}
repository.
\\\\
To run the test program :
\begin{minted}[linenos=true,
mathescape,
numbersep=5pt,
frame=single,
framesep=2mm
]{c++}
 meanShift - to run using camera input
 meanShiftTest {video file name} - to run using a video file
\end{minted}
Select the region of interest and click the build model
button to start tracking.
\\\\
For video file initially only the first frame is show ,select
the ROI in the first frame and then click on build model button
to start the tracking.
\\\\
The button is shown upon clicking on \texttt{Display properties}
button on the window.



% \subsection{Tracking using Similarity}
% Let us assume that the scene is static and our estimate of object location is at some
% point $(y)$ in the image.
% \\\\
% The tracking problem can be posed as similarity search problem.
% The color probability distribution is calculated in Region of interest about location in current frame and search is performed for region with color probability
% distribution in the next frame which is most similar to region in the current frame.
% \\\\
% If we estimate the object location correctly will get a similarity measure of 1.
% \\\\
% Let us assume that object was located at y,and now moved a small distance $h$
% If we compute the histogram over a region centered at y ,we will get a similarity 
% that is less 1.
% \\\\
% We need to move to a location such that maximum similarity is observed.
% \\\\
% Thus we need to find the location corresponding to modes of similarity function
% \section{Similarity function}
% Let us say we have two histograms,an object and the target histogram we can compute the similarity measure
% between the object model and the target at location y.
% \\\\
% Bhattacharrya coefficients provides a similarity measure between histograms.
% Thus bhattacharrya coefficient between object model q and color p of target at y is given by
%  \begin{equation}
%   \begin{array}{lcr}
%   p(y) = (p_1(y),\ldots,p_m(y)) \\
%   q    = (q_1,\ldots,q_m) \\
%   \rho (p(y),q)= \sum_{u=1}^{M} \sqrt{p_u(y)q_u} \\
%   \end{array}
%   \end{equation}
% Larger the value $\rho$ ,better is the match.
% \\\\
% For a perfect match similarity 1,and perfect mismatch 0.
% \subsection{Modes of similarity function}
% Using taylors series approximations
%  \begin{equation}
%   \begin{array}{lcr}
%   \rho(p(z),q)= \sum_{u=1}^{M} \sqrt{p_u(z=y+h)q_u} \\
%   = \sum_{u=1}^{M} \sqrt{p_u(y)q_u} + p_u'(y)h*0.5\sqrt{\frac{q_u}{p_u(y)}} \\
%   =\sum_{u=1}^{M} 0.5 \sqrt{p_u(y)*q_u}+0.5*\sqrt{p_u(y)*q_u}+p_u'(y)h* \sqrt{\frac{q_u}{p_u(y)}} \\
%   =\sum_{u=1}^{M} 0.5 \sqrt{p_u(y)*q_u} + 0.5* (p_u(y) + p_u'(y)h)  * \sqrt{\frac{q_u}{p_u(y)}} \\
%   =\sum_{u=1}^{M} 0.5 \sqrt{p_u(y)*q_u} + 0.5* (p_u(z))  * \sqrt{\frac{q_u}{p_u(y)}} \\
%   %\text{large $\rho$ denotes a good match}
%   \end{array}
%   \end{equation}
%   Where $p_u(y)$ denotes the target distribution about region centered at y
%   and $p_u(z)$ denotes the target distribution about region centered at z.
%   \\\\
% The first term in the summation is independent of z,the second term in summation is dependent on z.
% To find the object location we can maximize the similarity $\rho(p(z),q)$ wrt z,by maximizing the second
% term $\sum_{u=1}^M (p_u(z))  * \sqrt{\frac{q_u}{p_u(y)}} $
% \\\\
% Finding the modes of this function will give the modes of similarity function
% \\\\
% y is location of object in the previous frame,location of object to be estimated for maximum similarity
% Thus we can find the value of z ,which maximizes the similarity function.
% \\\\
% $(p(z))  * \sqrt{\frac{q}{p(y)}}$ is a weighted PDF.
% $p_u(z)$ can be estimated using kernel density estimate.\\\\
% However  $p_u(z)$ is weighted by $\sqrt{\frac{q_u}{p_u(y)}}$
% \\\\
% $q_u$ is fully specified and is known,$p_u(y)$ is the target histgram
% computed over a region of interest at current location,which again is known
% \\\\
% Thus ratio can be computed for region of interest about current location $y$
% \\\\
% \begin{itemize}
% \item It takes a value of 1
% if probability of pixel observed in object model is same as target model at location $y$.
% \item It takes a value greater than 1 is probability is reduced in target distribution
% \item It takes value lower than 1 if probability is higher is the target distribution
% \item It takes a value of 0 for pixels not observed in the object model.
% \end{itemize}
% The term is like a weighted mean,higher weights are assigned to pixels/bins
% which have a lower probability of being observed in target histogram ,indicating
% a mismatch.
% \\\\
% value of object model which are dominant are weighted higher if the observation in target
% model is lower.Lower weights are assignet to pixels/bins having higher probability than object model
% indiating good match for these pixels/bins
% \\\\
% Kernel density estimation encorporates a dynamic weighting scheme based on the similarity
% between object and target model.
% \\\\
% \subsection{Kernel Density Estimation for similarity function}
% If we assumed that the obtained similarity function is a PDF,and we sampled from the PDF
% higher density of points  would be obtained near the modes of this function
% \\\\
% Further each of these points can be associated with a weight 
% \begin{eqnarray}
%  p(z) = C_h \sum k( |\frac{z - x_i}{h}|^2 )
%  p_u(z) = C_h \sum_{i} w_i 
%  
% \end{eqnarray}
% 
% \begin{eqnarray}
%  
% \end{eqnarray}
% 
% 
% \section{Mean Shift Tracking}
% Mean shift is a algorithm for find the modes of distribution from a set of data points
% sampled from underlying unknown distribution.
% \\\\
% Large density of points are located at region where PDF has high value
% and density of points is smaller at regions where PDF has low value.
% \\\\
% In mean shift algorithm we were trying to find the location at which PDF exhibits maximum values.
% The parameters were the x and y co-ordinates
% \\\\
% Thus if we have 
% 
% 
% Given a initial location of object,the appearence of object is modelled using
% its color probability histogram in HS color space.
% \\\\
% given two histogram we can determine their similarity.A naive approach
% to color based tracking would be search in all possible locations
% about the current point where maximum object histogram similarity can
% be observed.
% \\\\
% However potential locations to search are large even if we assume small
% motion.
% \\\\
% The aim is to find the location where similarity function  is maximum.
% The similarity function assigns a value between 0 and 1 ,by considing
% a fixed region about the current point and computing the target model and
% them comparing this with the object model.
% \\\\
% Some points will exhibit large similarity
% If the similarity measure is smooth ,we can use the mean shift algorithm
% to find the modes of the similarity function.
% \\\\

% \section{Mean Shift Tracking}
% Mean shift tracking can be used for color based tracer.
% 
% 
% 
% 
% 
% \subsection{Constructing the object model}
% Let us consider that object is centered at location $y$ and we have to construct the object model
% in the form of histogram or probability distribution $p_u$,which provide the proability
% that pixel intensity at location belongs to the color/intensity u.
% \\\\
% \[
%   p_u = C \sum_{i=1}^n \delta(b(y_i)-u) \\\\
% \]
% where $b(y_i)$ denotes the bin corresponding to the pixel intensity at $y_i$
% and $p_u(y)$ denotes the probability that pixel intensity at y belongs to color $u$.
% \\\\
% 
% \subsection{Likelyhood image}
% Now we have a object model,given an image we can compute the likeyhood image
% Each pixel in likeyhood image represents the llikyhood that pixel belongs
% to the object model/histogram.
% \[
%  L_u(y_i) = p_u*\delta(b(y_i)-u)
% \]
% If we compute the mean of likelyhood image,it would give us the best estimate
% of location of object center $y$.
% \begin{figure}[!htbp]
% \begin{subfigure}[b]{0.3\textwidth}
%                 \centering
%                 \includegraphics[width=\textwidth]{image016.png}
%                 \caption{original image}                
%         \end{subfigure}%
%   \begin{subfigure}[b]{0.3\textwidth}
%                 \centering
%                 \includegraphics[width=\textwidth]{image015.png}
%                 \caption{likelyhood image}                
%         \end{subfigure}%
%   \begin{subfigure}[b]{0.3\textwidth}
%                 \centering
%                 \includegraphics[width=\textwidth]{image017.png}
%                 \caption{Hue histogram}                
%         \end{subfigure}%
% 
%         \caption{Object model}\label{fig:image1}
% \end{figure}
% 
% 
% 
% \subsection{Hitsotgram Similarity}
% Let us say we have two histograms,an object and the target histogram we can compute the similarity measure
% between the object model and the target at location y.
% \\\\
% Bhattacharrya coefficients provides a similarity measure between histograms.
% Thus bhattacharrya coefficient between object model q and color p of target at y is given by
%  \begin{equation}
%   \begin{array}{lcr}
%   p(y) = (p_1(y),\ldots,p_m(y)) \\
%   q    = (q_1,\ldots,q_m) \\
%   \rho (p(y),q)= \sum_{u=1}^{M} \sqrt{p_u(y)q_u} \\
%   \end{array}
%   \end{equation}
% Larger the value $\rho$ ,better is the match.
% \\\\
% For a perfect match similarity 1,and perfect mismatch 0.
% \\\\
% \begin{figure}[!htbp]
% \begin{subfigure}[b]{0.5\textwidth}
%                 \centering
%                 \includegraphics[width=\textwidth]{image018.png}
%                 \caption{displaced image similarity 1}                
%         \end{subfigure}%
%   \begin{subfigure}[b]{0.5\textwidth}
%                 \centering
%                 \includegraphics[width=\textwidth]{image019.png}
%                 \caption{displaced image similarity 2}                
%         \end{subfigure}%
%         \caption{Object model}\label{fig:image1}
% \end{figure}
% The first image in figure \ref{fig:image1} is used to build the histogram and shows a high similarity of
% almost one .The second image which is displaced by some ammount shows a lower similarity of 0.83.
% \\\\
% Using taylors series approximations
%  \begin{equation}
%   \begin{array}{lcr}
%   \rho(p(z),q)= \sum_{u=1}^{M} \sqrt{p_u(z=y+h)q_u} \\
%   = \sum_{u=1}^{M} \sqrt{p_u(y)q_u} + p_u'(y)h*0.5\sqrt{\frac{q_u}{p_u(y)}} \\
%   =\sum_{u=1}^{M} 0.5 \sqrt{p_u(y)*q_u}+0.5*\sqrt{p_u(y)*q_u}+p_u'(y)h* \sqrt{\frac{q_u}{p_u(y)}} \\
%   =\sum_{u=1}^{M} 0.5 \sqrt{p_u(y)*q_u} + 0.5* (p_u(y) + p_u'(y)h)  * \sqrt{\frac{q_u}{p_u(y)}} \\
%   =\sum_{u=1}^{M} 0.5 \sqrt{p_u(y)*q_u} + 0.5* (p_u(z))  * \sqrt{\frac{q_u}{p_u(y)}} \\
%   %\text{large $\rho$ denotes a good match}
%   \end{array}
%   \end{equation}
%   Where $p_u(y)$ denotes the target distribution about region centered at y
%   and $p_u(z)$ denotes the target distribution about region centered at z.
%   \\\\
% The first term in the summation is independent of z,the second term in summation is dependent on z.
% To find the object location we can maximize the similarity $\rho(p(z),q)$ wrt z,by maximizing the second
% term $\sum_{u=1}^M (p_u(z))  * \sqrt{\frac{q_u}{p_u(y)}} $
% \\\\
% The terms is the ratio of histograms/likelyhoods.
% \begin{itemize}
%  \item It takes a value of 1
% if probability of pixel observed in object model is same as target model at location $y$.
% \item It takes a value greater than 1 is probability is reduced in target distribution
% \item It takes value lower than 1 if probability is higher is the target distribution
% \item It takes a value of 0 for pixels not observed in the object model.
% \end{itemize}
% The term is like a weighted mean,higher weights are assigned to pixels/bins
% which have a lower probability of being observed in target histogram ,indicating
% a mismatch.
% \\\\
% Lower weights are assignet to pixels/bins having higher probability than object model
% indiating good match for these pixels/bins
% \\\\
% Pixels not in object model are not considered in computing the mean.
% \\\\
% 
% 
% 
% 
% Let us assume at the initial position $q_u=p_u(y)$ for all u
% \\\\
% Thus we are left to maximize $\sum_{u=1}^M (p_u(z))$
% \\\\
% Value of z which gives the highest value of mean of target proability distribution is 
% best estimate of object location.
% \\\\
% If we set $z=m(z)$ we have estimated the object location.
% 
% 
% 
% \section{Tracking}
% Object is tracked in given frame by matching color probability of target with that of object model.
% \\\\
% This probability map/color probabilit histogram serves as primary input to the mean shift tracked along with the location
% estimate of object.Typically the initial location estimate is provided by the user by selecting the ROI.
% \\\\
% Mean shift tracking assumes that object does not move drastically.Given frames $\mathcal{I}$ and $\mathcal{J}$
% and ROI $\mathcal{R}$ defined in the frame $\mathcal{I}$ it is assumed that some part of the object is visible
% in the ROI defined by $\mathcal{R}$ in the frame $\mathcal{J}$.
% \\\\
% Thus color probability distribution about region $\mathcal{R}$ in frame $\mathcal{I}$ will bear some similarity
% with the color probability distribution about region $\mathcal{R}$ in frame $\mathcal{J}$.
% \\\\
% To determine the location of the object the similarity measure needs to be translated into spatial information.
% The algorithm should estimate the location where maximum similarity is observed .
% \\\\
% The best candidate location is found by maximizing a similarity function.
% \\\\
% In the figure \ref{fig:image1} ,is we are able to move to a position that has highest similarity locally
% then we can assume that the object has been tracked successfully.
% \\\\
% 
% 
% 
% If Similarity function is convex,smooth and differentiable,we can find a way to translate
% the similarity information to spatial information.
% \\\\
% 
% 
% \section{Mean Shift Algorithm}
% Mean shift algorithm forms the basis for the tracking algorithm.
% \\\\
% Brute force approach can be to perform the similarity comparision over some neighborhood
% of current location and choose new estimate as location which gives the maximum similarity
% \\\\
% However if we assume that new location z is located near to y and color probability
% does not change drastically
% \\\\
% Using binomial theorem
% 
% 
% Let us assume that object has moved and we estimate new location $\bar{y}$
% \\\\
% The sample mean $m$ at $\bar{y}$ with kernel K or the weighted mean is given by
%  \begin{equation}
%   \begin{array}{lcr}
%     m(x) = \frac{\sum_{i=1}^\mathcal{N} K(\bar{y}-x_i)y_i}{\sum_{i=1}^{\mathcal{N}} K(\bar{y}-y_i)}
%   \end{array}  
%   \end{equation}      
% \\\\
% 
% 
% 
% If we did not known y ,but have a set of point $y_i$,using the above formulation
% we can estimate the value $y$ that gives the maximum probability for specific color $u$.
% \\\\
% 
% 
% 
% Consider a Set $\mathcal{S}$ of $\mathcal{N}$ data points $x_i$ in 2-D Euclidean space $\mathcal{X}$.
% \\\\
% Let $K(x)$ denote a kernel function that indicates weighted contribution of point $x_i$ to estimation of the mean.
% The difference $m(x) - x$ is called the mean shift.
% \\\\
% The aim of the mean shift algorithm is to iteratively move the data point to its mean.In each iteration
% $x \leftarrow m(x)$.The algorithm stops when $m(x)=x$.
% \\\\
% The sequence $x,m(x),m(m(x)) \ldots$ indicates  trajectory iteration takes to reach x.
% \\\\
% \subsection{Kernel function}
% Typically Kernel is is a function of $||x||^2$. $K(x) = k(||x||^2)$
% and k is called the profile of the kernel.
% Some commom properties of profile are
% \begin{itemize}
%  \item k is non negative
%  \item k is non increasing 
%  \item k is piecewise continuous and has finite area.
% \end{itemize}
% The aim of the kernel is to given importance to data points near the point at which mean is estimated
% than other points.
% \\\\
% Commonly used kernels are flat and gaussian kernels.In the present implementation we will use a flat kernel/
% \subsection{Kernel Density Estimation}
% Kernel density estimation is a popular method of estimating the probability distribution.
% \\\\
% For a set of $\mathcal{N}$ data points $x_i$ in 2-D space the kernel density estimate with kernel K
% and radius h is given by
%  \begin{equation}
%   \begin{array}{lcr}
%   \bar{f}_K(x) = \frac{1}{nh^2}\sum_{i=1}^{\mathcal{N}} K(\frac{x-x_i}{h})
%   \bar{f}_K(x) = \frac{1}{nh^2}\sum_{i=1}^{\mathcal{N}} k(||\frac{x-x_i}{h}||^2)
%   \end{array}
%   \end{equation}
% The quality of estimation is measured as mean square error between the actual density and the estimat.
% \\\\
% Define a kernel $G(x) = g(||x||^2)$ such that
%  \begin{equation}
%   \begin{array}{lcr}
%   g(x) = - k'(x) = - \frac{\partial dk(x)}{\partial x}
%   \end{array}
%   \end{equation}
% \\\\
% since kernel function is used to model probability density function.$f_K(x)$ represents the statistical mean
% of the probability distribution modelled by the kernel function.
% \\\\
% \subsection{Mean shift}
% Mean shift with kernel G will move along the direction of gradient of density estimate $\bar{f}$ with kernel K.
% This is shown below.
% \\\\
% Define a density estimate with K ,perform mean with G.Mean shift will perform gradient ascent on density estimate.
% \\\\
%  \begin{equation}
%   \begin{array}{lcr}
%   \text{Consider density estimate with kernel G}\\\\
%   \bar{f}_{G}(x) = \frac{C}{nh^2}\sum_{i=1}^{\mathcal{N}} G(\frac{x-x_i}{h})\\
%   \bar{f}_{G}(x) = \frac{C}{nh^2}\sum_{i=1}^{\mathcal{N}} g(||\frac{x-x_i}{h}||^2)
%   \\
%   \text{where C is a normalization constant}
%   \\\\
%   \text{Mean shift with kernel G is}\\
%   M_{G}(x) = \frac{\sum_{i=1}^\mathcal{N} g(||\frac{x-x_i}{h}||^2)x_i}{\sum_{i=1}^{\mathcal{N}} g(||\frac{x-x_i}{h}||^2)}-x\\
%   \text{Estimate of density gradient is gradient of density estimate}
%   \bar{\nabla} f_K(x) 	= \nabla \bar{f}_K(x)\\
%   \bar{\nabla} f_K(x)	=\frac{2}{nh^4}\sum_{i=1}^{\mathcal{N}} (x-x_i)k'(||\frac{x-x_i}{h}||^2)\\
%   \bar{\nabla} f_K(x)	=\frac{2}{nh^4}\sum_{i=1}^{\mathcal{N}} (x-x_i)g(||\frac{x-x_i}{h}||^2)\\
%   \bar{\nabla} f_K(x)	=\frac{2}{nh^4}\sum_{i=1}^{\mathcal{N}} (x-x_i)g(||\frac{x-x_i}{h}||^2)\\
%   \bar{\nabla} f_K(x)	=\frac{2}{nh^4}\sum_{i=1}^{\mathcal{N}} (x-x_i)g(||\frac{x-x_i}{h}||^2)\\
%   \bar{\nabla} f_K(x)	=\frac{2}{nh^4}\sum_{i=1}^{\mathcal{N}} g(||\frac{x-x_i}{h}||^2) * M_{G}(x)\\
%   \bar{\nabla} f_K(x)	=\frac{2}{Ch^2}\bar{f}_{G}(x) * M_{G}(x)\\
%   M_{G}(x) = 0.5*Ch^2 \frac{\bar{\nabla} f_K(x)}{\bar{f}_{G}(x)}\\  
%   \end{array}
%   \end{equation}
% This proves $M_{G}(x)$ provides an estimate of normalized gradient of $f_{K}$.
% Thus as we apply mean shift tracking with kernel G we will obtain estimate of the normalized gradient
% of $f_{k}$.If location has been reached such that $m_g(x)=x$ then mean shift will be zero indicating
% the normalized gradient values is zero  and thus point at which the density is maximum has been reached.
% \\\\
% 
% \subsection{Similarity measure}
% % The mean shift will be zero if the similarity between the object model and current frame at specified
% % location is maximum.Thus indicating location of point at which maximum model probability is observed
% % as been reached.
% % \\\\
% In the present implementation location where the maximim similarity wrt object is observed is determined.
% \begin{itemize}
%  \item Let $x_i,i=1,2,\dots,N $ denote pixels locations of model centered at 0.
%  \item Let the color distribution be represented by m color histogram bin
%  \item Let $b(x_i)$ denote the color of the bin at location $x_i$
%  \item Assume the model is normalized and kernel radius is 1.
% \end{itemize}
% The probability q of color u in object model can b represented as 
%  \begin{equation}
%   \begin{array}{lcr}
%     q_u = C \sum_{i=1}{N}k(||x_i||^2)\delta(b(x_i) - u_)\\
%     C=[\sum_{i=1}{N}k(||x_i||^2)]^-2\\
%     \text{Kernel weights contribution of point by distance to the centroid}\\
%     \text{Let $y_i$ denote target location of pixels centered at y}\\
%     \text{The probabililty p of color u in target candidate is given by}\\
%     p_u = C_h \sum_{i=1}{N_h}k(||\frac{y-y_i}{h}||^2)\delta(b(y_i) - u_)\\
%     C_h=[\sum_{i=1}{N}k(||\frac{y-y_i}{h}||^2]^-2\\    
%   \end{array}
%   \end{equation}
% % Thus we have the object and the target histogram we can compute the similarity measure
% % between the object model and the target at location y.
% % \\\\
% % Bhattacharrya coefficients provides a similarity measure between histograms.
% % Thus bhattacharrya coefficient between object model q and color p of target at y is given by
% %  \begin{equation}
% %   \begin{array}{lcr}
% %   \rho(p(y),q)= \sum_{u=1}^{M} \sqrt(p_u(y)q_u)
% %   \text{large $\rho$ denotes a good match}
% %   \end{array}
% %   \end{equation}
% Let y denote the current target location and Let z denote estimated new target location near y .
% \\\\
% And since color probability will not change drastically between adjacent frames the similarity coefficient
% can be approximated using first order taylor's polynomial.
% \\\\
%  \begin{equation}
%   \begin{array}{lcr}
%   \rho(p(z),q)= 0.5*\rho(p(y),q) + 0.5 \sum_{u=1}^{M}p_u(z)\sqrt(\frac{q_u}{p_u(y)})
%     \text{Probaiblity p of color u in the target candidate centered as z}
%     p_u(z) = C_h \sum_{i=1}^{N_h}k(||\frac{z-y_i}{h}||^2)\delta(b(y_i) - u_)\\\\
%     rho(p(z),q)= 0.5*\rho(p(y),q) + 0.5 \sum_{u=1}^{M}C_h \sum_{i=1}^{N_h}k(||\frac{z-y_i}{h}||^2)\delta(b(y_i) - u_) \sqrt(\frac{q_u}{p_u(y)})\\\\
%     rho(p(z),q)= 0.5*\rho(p(y),q) + \frac{C_h}{2} \sum_{i=1}^{N_h} \sum_{u=1}^{M} (\delta(b(y_i) - u_) \sqrt(\frac{q_u}{p_u(y)}))	k(||\frac{z-y_i}{h}||^2)\\\\
%     rho(p(z),q)= 0.5*\rho(p(y),q) + \frac{C_h}{2} \sum_{i=1}^{N_h} w_i	k(||\frac{z-y_i}{h}||^2)\\\\
%     \text{First term is independent of z ,second term is dependent of z}\\
%     \text{To maximize similarity maximize the second term wrt z}\\        
%     \frac{C_h}{h} \sum_{i=1}^{N_h} w_i	(z-y_i) k'(||\frac{z-y_i}{h}||^2) =0 \\\\
%     \sum_{i=1}^{N_h} w_i y_i k'(||\frac{z-y_i}{h}||^2) = \sum_{i=1}^{N_h} w_i z (||\frac{z-y_i}{h}||^2)\\\\
%     z = \sum_{i=1}^{N_h} w_i y_i g(||\frac{z-y_i}{h}||^2)  / \sum_{i=1}^{N_h} w_i g(||\frac{z-y_i}{h}||^2)\\\\
%     \text{Since the color probability is assumed to not change drasitically}\\
%     z = \sum_{i=1}^{N_h} w_i y_i g(||\frac{y-y_i}{h}||^2)  / \sum_{i=1}^{N_h} w_i g(||\frac{y-y_i}{h}||^2)\\\\    
%   \end{array}
%   \end{equation}
%   Thus estimate of location which provides the maximum similarity is obtained as weighted summation
%   of kernel function $G_(x)$.   
% \section{Mean Shift Tracking Algorithm}
% given probability q of set of colors ${q_u}$ of model and location y in the previous frame
% \begin{itemize}
%  \item Initialize the location of the target in the current frame as y
%  \item compute the probability p of set of colors ${u}$ in the current frame $p_u(y)$ and similarity coefficient $\rho(p(y),q)$
%  \item compute the weights $w_i$ over the region in the current frame.
%  \item Compute the new location of z 
%  \item compute the probability $p_u(z)$ and similarity measure $\rho(p(y),q)$
%  \item if $\rho(p(z),q)$ < $\rho(p(y),q)$ do $z\leftarrow0.5*(z+y)$.
%  \item if $|z-y|$ is small stop else $y\leftarrow z$
% \end{itemize}
% \section{Implementation Details}
% In the present implementation kernel function used is a flat kernel.
% \\
% The object model used is the color probability histogram in the HS channels of HSV color space.
% \\
% Initial location and ROI is provided as input to the algorithm.
% \\
% The color probability distribution can be calculated from the provided ROI.
% \\
% Thus at initialization a probability q of set of color ${q_u}$ of model and location is available y.
% \\
% With the same location y and the given ROI compute the color probability distribution of next frame.
% \\
% since we have color probability distribution is the current and the next frame the similarity coefficient can be evaluated.
% \\
% Considering a flat kernel.
% \\
%  \begin{equation}
%   \begin{array}{lcr}
%     g(x)=-k'(x) \\
%     g(x)=-\frac{\partial (x-x_i)/h}{\partial x} \\
%     g(x)= \frac{x_i}{h};
%     g(x)=x_i        
%     z = \sum_{i=1}^{N_h} w_i)y_i*y_i||^2)  / \sum_{i=1}^{N_h} w_i*y_i\\\\    
%     w_i =\sum_{u=1}^{M} (\delta(b(y_i) - u_) \sqrt(\frac{q_u}{p_u(y)}))\\\
%   \end{array}
%   \end{equation}


\end{document}